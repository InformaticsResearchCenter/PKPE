\chapter{Standar Perlengkapan}
\textit{Error} terjadi pada suatu program akibat ketidaksesuaian penyusunan suatu program dengan standar yang sudah ditetepkan.
Dalam kasus penangan \textit{error}, beberapa bahasa pemrograman memiliki \textit{IDE} yang merupakan singkatan dari \textit{Integrated Development Environment} yang dapat melakukan pengecekan \textit{error} secara \textit{realtime}. Disini kita membutuhkan \textit{IDE} dari suatu bahasa pemrograman yang memiliki \textit{variable explorer} . \textit{variable explorer} berfungsi menampilkan konten-konten apasaja yang ada di baris \textit{coding}-an kita, yang bertujuan untuk mempermudah kita untuk menyusun \textit{code} program. Jadi perlengkapan yang kita harus persiapkan adalah :

\begin{enumerate}
\item Bahasa pemorgraman
\item \textit{IDE} dari suatu bahasa pemrograman yang memiliki \textit{variable explorer}.
\end{enumerate}

\section{Jenis \textit{Error}}
\par 
\textit{Error} atau bisa disebut dengan kesalahan pada program memiliki berbagai jenis tipe, diantaranya :

\begin{enumerate}
\item 
\textit{Syntax errors}
\par 
\textit{Syntax errors} adalah \textit{error} yang diakibatkan oleh kesalahan dalam penulisan bahasa program yang tidak dapat dimengerti oleh \textit{compiler} , contohnya dapat dilihat pada listing \ref{lst:1} berikut.

\begin{lstlisting}[language=C, caption=Contoh Syntax errors,label={lst:1}]
#include <stdio.h>

int main(void) {
    printf("Poltekpos\n";
    return 0;
}
\end{lstlisting}
 
Terdapat \textit{syntax error} pada program ini, \textit{syntax error} disini diakibatkan kurangnya tanda ")" pada akhir baris kode \textbf{ke-4}. Kurangnya "\;" , ")" , "]" , "\}" pada akhir atau awal baris \textit{code} juga dapat menyebabkan terjadinya \textit{syntax error}.

\item 
\textit{Semantic errors}
\par 
\textit{Semantic errors} terjadi akibat tidak tepatnya varibel dengan \textit{statement} yang sudah dibuat, ketidakjelasan logika pada program yang dibuat akan menimbulkan \textit{semantic error} contohnya dapat dilihat pada listing \ref{lst:2} berikut.

\begin{lstlisting}[language=Java,caption=Contoh \textit{Semantic errors},label={lst:2}]
public static void main(String[] args) {	
		Int NPM;		
		NPM = "Rayhan";
		System.out.println(NPM);
	}
\end{lstlisting}
\par 
Program ini akan menghasilkan \textit{semantic error},\textit{compiler} tetap dapat menjalan kan program tersebut,namun output yang dikeluarkan berupa String yang terapat pada baris ke\textbf{3}. jika output yang diinginkan berupa NPM yang bertipe data INT sperti pada baris ke\textbf{2}, maka tipe data yang diinputkan harus sesuai dengan output yang diinginkan.
\end{enumerate}